\chapter{Conclusions and Future Directions}
\label{sec:conclusions}

In this thesis we have introduced new building blocks and
fundamental components for machine learning that enable
optimization-based domain knowledge to be injected
into the modeling pipeline.
We have presented the \emph{OptNet} architecture as a
foundation for convex optimization layers and the
\emph{input-convex neural network} architecture as a
foundation for deep convex energy-based learning.
We have shown how these techniques can be applied to
differentiable model-predictive control and
top-$k$ learning.
Differentiable optimization-based modeling components
provide an expressive set of operations and
have a promising set of future directions.
To enable rapid prototyping in this space, we have shown
how \cvxpy can be turned into a differentiable layer.
In the following we discuss areas where optimization-based
has been applied and has potential to continue making
an impact.

\begin{itemize}
\item \textbf{Game theory.}
  The game theory literature typically focuses on finding
  optimal strategies of playing games with known rules.
  While the rules of a lot of games are known explicitly,
  scenarios could come up where it's useful to learn the
  rules of a game and to have a ``game theory''
  equilibrium-finding layer.
  For example in reinforcement learning, an agent can have an
  arbitrary differentiable ``game theory'' layer that is able
  of representing complex tasks, state spaces, and
  problems as an equilibrium-finding problem in a game
  where the rules are automatically extracted.
  This has started to be explored in
  \citet{ling2018game}.
\item \textbf{Stochastic optimization and end-to-end learning.}
  Typically probabilistic models are used in the context of
  larger systems. When these systems have an objective
  that is being optimized, it is usually ideal to incorporate
  the knowledge of this objective into the probabilistic modeling
  component.
  If the downstream systems involve solving
  stochastic optimization problems, as in power-systems,
  creating an end-to-end differentiable architecture is
  more difficult and can be done by using
  differentiable optimization as in \citet{donti2017task}.
\item \textbf{Reinforcement learning and control.}
  \begin{itemize}
  \item \textbf{Safety.} RL agents may be deployed in scenarios when
    the agent should avoid parts of the state space,
    \eg in safety-critical environments.
    Differentiable optimization layers can be used
    to help constrain the policy class so that these
    regions are avoided.
    This is starting to be explored in
    \citet{dalal2018safe,pham2018optlayer}.
  \item \textbf{Differentiable control and planning.}
    The differentiable MPC approach we presented in \cref{sec:empc}
    is just one step towards a significantly broader vision
    of integrating control and learning for doing imitation
    or policy learning.
  \item \textbf{Physics-based modeling.}
    When RL environments involve physical systems,
    it may be useful to have a physics-based modeling.
    This can be done with a differentiable
    physics engines as in \citet{de2018end}.
  \item \textbf{Inverse cost and reward learning.}
    In many multi-agent control scenarios modeling agents
    as controllers that are optimizing a control objective
    is a powerful paradigm \citep{ng2000algorithms,finn2016guided}.
    TODO
    Differentiable controllers
    are useful when trying to reconstruct an optimization
    problem that other agents are solving.
    This is done in the context of cost shaping in
    \citet{tamar2017learning}.
  \item \textbf{Multi-agent environments.} TODO
  \item \textbf{Control in high-dimensional state spaces.} TODO
  % \item Universal planning networks: \citet{srinivas2018universal}
  \end{itemize}
\item \textbf{Discrete, combinatorial, and submodular optimization.}
  The space of discrete, combinatorial, and mixed optimization problems
  captures an even more expressive set of operations than
  the continuous and convex optimization problems we have considered
  in this thesis.
  Similar optimization components can be made for some of these
  types of problems, as explored in
  \citet{djolonga2017differentiable,tschiatschek2018differentiable,mensch2018differentiable,niculae2018sparsemap,niculae2017regularized}.
\item \textbf{Meta-learning.}
  Some meta-learning formulations such as \citet{finn2017model}
  and \citet{ravi2016optimization}
  involve learning through an unrolled optimizer that
  typically solve an unconstrained, continuous, and
  non-convex optimization problem.
  In some cases, unrolling through a solver with
  many iterations may require inefficient amounts
  of compute or memory.
  Meta-learning methods could be improved by using a differentiable
  closed-form solver as
  \citet{bertinetto2018meta} explores.
\item \textbf{Optimization viewpoints of standard components.}
  A motivation behind this thesis work has been the optimization
  viewpoints of standard layers we discussed in \cref{sec:bg:existing}.
  Many other directions can be taken with the viewpoints, such
  as the proximal operator viewpoint in \citet{bibi2018deep}
  that interprets deep layers as stochastic solvers.
\end{itemize}

%%% Local Variables:
%%% coding: utf-8
%%% mode: latex
%%% TeX-engine: xetex
%%% TeX-master: "../thesis"
%%% End: