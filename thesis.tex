\PassOptionsToPackage{svgnames,dvipsnames}{xcolor}

\documentclass[12pt]{cmuthesis}

\usepackage[Lenny]{fncychap}
\ChNameVar{\Large}

\input{sections/packages}
\input{sections/macros}

% \includeonly{sections/cvxpyth}
\draftstamp{\today}{DRAFT}

\begin {document}
\frontmatter

\pagestyle{empty}

\title{{\bf Differentiable Optimization-Based Modeling for Machine Learning}}
\author{Brandon Amos}
\date{May 2019}
\Year{2019}
\trnumber{CMU-CS-19-X}

\committee{
\begin{tabular}{rl}
J. Zico Kolter, Chair & \textit{Carnegie Mellon University} \\
Barnab{\'a}s P{\'o}czos & \textit{Carnegie Mellon University} \\
Jeff Schneider & \textit{Carnegie Mellon University} \\
Nando de Freitas & \textit{DeepMind} \\
Vladlen Koltun & \textit{Intel Labs} \\
\end{tabular}
}

\support{
  % This thesis was supported by the National Science Foundation Graduate
  % Research Fellowship Program under Grant No.~DGE1252522.
}
\disclaimer{}

\keywords{machine learning, statistical modeling,
  convex optimization, deep learning, control,
  reinforcement learning}

\maketitle

\begin{dedication}
  To all of the people that light up my life. {\ensuremath\heartsuit}
\end{dedication}

\begin{abstract}
  Domain-specific modeling priors and specialized components are
  becoming increasingly important to the machine learning field.
  These components integrate specialized knowledge that we have
  as humans into model.
  We argue in this thesis that optimization methods provide an
  expressive set of operations that should be part of the
  machine learning practitioner's modeling toolbox.

  We present two foundational approaches for optimization-based modeling:
  1) the \emph{OptNet} architecture that integrates
  optimization problems as individual layers in larger end-to-end
  trainable deep networks, and
  2) the \emph{input-convex neural network (ICNN)}
  architecture that helps make inference and learning in deep
  energy-based models and structured prediction more tractable.

  We then show how to use the OptNet approach
  1) as a way of combining model-free and model-based reinforcement
  learning and
  2) for top-$k$ learning problems.
  We conclude by showing how to differentiate cone programs
  and turn the \cvxpy domain specific language into
  a differentiable optimization layer that enables rapid prototyping of
  the approaches in this thesis.
\end{abstract}

% \newgeometry{left=0.5in,right=0.5in,top=1in,bottom=1.4in}
\begin{acknowledgments}
  Conditional on this thesis being accepted I would like to
  make the following incomplete and unfinished
  set of acknowledgments. \vspace{6mm}

  I have been incredibly fortunate and privileged throughout
  my entire life to have been given many opportunities
  that have led me to pursue this thesis research.
  Thanks to the thousands of people in the universe throughout
  the past few millennia who have provided me with the
  foundation, environment, safety, health, support, service,
  financial well-being, love, joy, knowledge, kindness, calmness,
  and happiness to produce this work.

  This thesis would not have been possible without the close
  collaboration I have had with my advisor J.~Zico Kolter over
  the past few years.
  Zico's creativity and passion have profoundly shaped
  the way I think about academic problems and pursue
  research directions, and more broadly I have learned much
  more from him along the way.
  I am incredibly grateful for the immense
  amount of time and energy Zico has put into shaping the
  direction of this work and for molding me into who I am.

  Thanks to all of my close collaborators who have contributed
  to projects appearing in this thesis, including
  Byron Boots, Ivan Jimenez, Vladlen Koltun, Jacob Sacks, and Lei Xu,
  and more recently
  Akshay Agrawal,
  Shane Barratt,
  Stephen Boyd,
  Steven Diamond,
  and Brendan O'Donoghue.

  This thesis was also made possible by the great research
  environment that CMU has provided me during my studies here.
  CMU's collaborative, thriving, and understanding environment gave
  me the true capabilities to pursue my passions throughout my time here.
  I spent my first two years honing my systems skills working on
  wearable cognitive assistance applications with
  Mahadev (Satya) Satyanarayanan and am
  indebted to him for kindly giving me the freedom to pursue my
  interests in machine learning while part of his systems group.
  I hope that someday I will be able to pay this kindness forward.
  Thanks also to all of the administrative staff that have
  kept everything at CMU running smoothly, including
  Deb Cavlovich and Ann Stetser.
  I am also very thankful to Gaurav Manek for a well-engineered
  cluster setup that has made running and managing
  experiments effortless for the rest of us.
  And thanks to everybody else at CMU who have made
  graduate school incredibly enjoyable.
  These wonderful memories will stay with me for life.
  This includes
  Maruan Al-Shedivat,
  Alnur Ali,
  Filipe de Avila Belbute-Peres,
  Shaojie Bai,
  Sol Boucher,
  Noam Brown,
  Volkan Cirik,
  Dominic Chen,
  Zhuo Chen,
  Michael Coblenz,
  Jeremy Cohen,
  Jonathan Dinu,
  Priya Donti,
  Gabriele Farina,
  Benjamin Gilbert,
  Kiryong Ha,
  Jan Harkes,
  Wenlu Hu,
  Roger Iyengar,
  Christian Kroer,
  Jonathan Laurent,
  Jay-Yoon Lee,
  Lisa Lee,
  Chun Kai Ling,
  Stefan Muller,
  Vaishnavh Nagarajan,
  Vittorio Perera,
  Padmanabhan (Babu) Pillai,
  George Philipp,
  Aurick Qiao,
  Leslie Rice,
  Wolf Richter,
  Mel Roderick,
  Petar Stojanov,
  Dougal Sutherland,
  Junjue Wang,
  Phillip Wang,
  Po-Wei Wang,
  Josh Williams,
  Ezra Winston,
  Eric Wong,
  Han Zhao, and
  Xiao Zhang.

  My Ph.D.~would have been severely lacking without my internships
  at DeepMind in 2017 and Intel Labs in 2018.
  I learned how to craft large-scale reinforcement learning systems
  from Nando de Freitas and Misha Denil at DeepMind and
  about cutting-edge vision research from
  Vladlen Koltun at Intel Labs.
  Thank you all for hosting me.
  I am also grateful for all of the conversations and collaborations
  with the other interns and researchers in the industry as well,
  including
  Yannis Assael,
  David Budden,
  Serkan Cabi,
  Kris Cao,
  Chen Chen,
  Qifeng Chen,
  Yutian Chen,
  Mike Chrzanowski,
  Sergio Gomez Colmenarejo,
  Tim Cooijmans,
  Soham De,
  Laurent Dinh,
  Vincent Dumoulin,
  Tom Erez,
  Michael Figurnov,
  Jakob Foerster,
  Marco Fraccaro,
  Yaroslav Ganin,
  Katelyn Gao,
  Yang Gao,
  Caglar Gulcehre,
  Karol Hausman,
  Matthew W.~Hoffman,
  Drew Jaegle,
  David Lindell,
  Hanxiao Liu,
  Simon Kohl,
  Alistair Muldal,
  Alexander Novikov,
  Tom Le Paine,
  Ben Poole,
  Rene Ranftl,
  Scott Reed,
  German Ros,
  Evan Shelhamer,
  Sainbayar Sukhbaatar,
  Casper Kaae Sønderby,
  Brendan Shillingford,
  Yuval Tassa,
  Jonathan Uesato,
  Ziyu Wang,
  Abhay Yadav,
  Xuaner Zhang, and
  Yuke Zhu.

  I am grateful to the broader machine learning research community
  that has been thriving throughout my studies and has
  supported the direction of this work.
  This includes the Caffe, PyTorch, and TensorFlow communities
  I have interacted with over the years.
  These ecosystems have made the implementation and engineering
  side of this thesis easy and enjoyable.
  Thanks especially to Soumith Chintala, Adam Paszke, and the rest
  of the (Py)Torch community for helping me debug many strange
  errors and eventually contribute back.
  And thanks to everybody in the broader machine learning community
  who has given me deeper insights into problems or has graciously
  helped me with their code, including
  David Belanger,
  Alfredo Canziani,
  Alex Terenin, and
  Rowan Zellers.

  Thanks to all of the other communities that have provided me
  with the tooling and infrastructure necessary that allows
  me to work comfortably. These communities deserve more credit
  for the impacts that they have and the immense amount of
  development effort behind them and include the
  emacs \citep{stallman1981emacs},
  git \citep{torvalds2005git},
  hammerspoon,
  homebrew,
  \LaTeX \citep{lamport1994latex},
  Linux,
  mjolnir,
  mu4e,
  mutt,
  tmux,
  vim,
  xmonad \citep{stewart2007xmonad}, and
  zsh projects,
  as well as the many pieces of the Python ecosystem
  \citep{van1995python,oliphant2007python}, especially
  Jupyter \citep{kluyver2016jupyter},
  Matplotlib \citep{hunter2007matplotlib},
  seaborn,
  numpy \citep{van2011numpy},
  pandas \citep{mckinney2012python}, and
  SciPy \citep{jones2014scipy}.

  Looking back, my teachers and mentors earlier in my life
  ignited my interests in mathematics and computer science
  and opened my eyes.
  My high school teachers
  Suzanne Nicewonder,
  Susheela Shanta, and
  Janet Washington gave me a solid foundation
  in engineering and mathematics.
  Mack McGhee at Sunapsys hosted me for an
  internship that introduced to the wonderful
  world of Linux.
  Moving into my undergrad,
  Layne T.~Watson and David Easterling
  introduced me to the beautiful fields
  of optimization, numerical methods, and
  high-performance computing, and taught me how to
  write extremely optimized and robust Fortran code.
  I apologize for going to the dark side and writing
  ANTODL (another thesis on deep learning).
  Jules White and Hamilton Turner taught me how
  to hack Android internals and architect awesome Scala code.
  Binoy Ravindran, Alastair Murray, and Rob Lyerly
  taught me how to hack on compilers
  and the Linux kernel.

  On the personal side, I would like to thank all of my
  other friends, family members, and partners that
  have provided me with an immense amount of love,
  support, and encouragement throughout the years,
  especially Alice, Emma, and Nil-Jana.
  Thanks to my parents Sandy and David;
  brothers Chad and Chase;
  grandparents Candyth, Marshall, and Geneva;
  and the rest of my extended family
  for raising me in a wonderful environment and
  encouraging me at every step along the way.
  Thanks to my uncle Dan Dunlap for inspiring me and
  raving about AI, CS, philosophy, and music all of these years.
  And thanks to everybody else I have met in the
  arts,
  board games,
  climbing,
  cycling,
  dance,
  lifting,
  meditation,
  music,
  nature,
  poetry,
  theatre, and
  yoga
  communities in Pittsburgh, San Francisco, and London for
  providing a near-infinite amount of distractions from
  this thesis.
\end{acknowledgments}
% \restoregeometry

\pagestyle{plain}

\tableofcontents
\listoffigures
\listoftables
\listofalgorithms

\mainmatter

\include{sections/intro}
\chapter{Preliminaries and Background}
This section provides a broad overview of foundational ideas
and background material relevant to this thesis.
In most chapters of this thesis, we include a deeper
discussion of the related literature relevant to
that material.

\section{Preliminaries}
The content in this thesis builds on the following topics.
We assume preliminary knowledge of these topics and
give a limited set of key references here.
The reader should have an understanding of statistical
and machine learning modeling paradigms as described in
\citet{wasserman2013all,bishop2007pattern,friedman2001elements}.
Our contributions mostly focus on end-to-end modeling with
deep architectures as described in
\citet{schmidhuber2015deep,goodfellow2016deep} with
applications in computer vision as described in
\citet{forsyth2003modern,bishop2007pattern,szeliski2010computer}.
Our contributions also involve optimization theory and
applications as described in
\citet{bertsekas1999nonlinear,boyd2004convex,bonnans2013perturbation,griewank2008evaluating,nocedal2006sequential,sra2012optimization,wright1997primal}.
One application area of this thesis work focuses on control
and reinforcement learning.
Control is one kind of optimization-based modeling and
is further described in
\citet{bertsekas2005dynamic,sastry2011adaptive,levine2017optimal}.
Reinforcement learning methods are summarized in
\citet{sutton1998reinforcement,levine2017introduction}.

\section{Energy-based Learning}
Energy-based learning is a machine learning method
typically used in supervised settings that explicitly
adds relationships and dependencies to the model's
output space.
This is in contrast to purely feed-forward models that
typically cannot explicitly capture dependencies
in the output space.
At the core of energy-based learning methods is
a scalar-valued \emph{energy} function
$E_\theta(x,y): \mathcal{X}\times\mathcal{Y}\rightarrow \mathbb{R}$
parameterized by $\theta$ that measures the fit
between some input $x$ and output $y$.
Inference in energy-based models is done by
solving the optimization problem
\begin{equation}
  \label{eq:bg:inf}
  \hat y = \argmin_{y} E_\theta(x, y).
\end{equation}
We note that this is a powerful formulation for
modeling and learning and subsumes the representational
capacity of standard deep feedforward models,
which we show how to do in \cref{sec:bg:ff-energy}.
The energy function can also be interpreted
from a probabilistic lens as the negated unnormalized
joint distribution over the input and output spaces.

Energy-based methods have been in use for over a decade
and the tutorial \citet{lecun2006tutorial} overviews
many of the foundational methods and challenges
in energy-based learning.
The two main challenges for energy-based learning are
1) learning the parameters $\theta$ of the energy
function $E_\theta$ and 2) efficiently solving
the inference procedure in \cref{eq:bg:inf}.
These challenges have historically been tamed by
using simpler energy functions consisting of
hand-engineered feature extractors for the inputs
$x$ and linear functions of $y$.
This captures models such as Markov random fields
\citep{li1994markov}
and conditional random fields
\citep{lafferty2001conditional,sutton2012introduction}.
Standard gradient-based methods are difficult to use
for parameter learning because $\hat y$ depends on $\theta$
through the $\argmin$ operator, which is not always differentiable.
Historically, a common approach to doing parameter learning
in energy-based models has been to directly shape the
energy function with a max-margin approach
\citet{taskar2004max,taskar2005learning}.

More recently, there has been a strong push to further incorporate
structured prediction methods like conditional random fields as the
``last layer'' of a deep network architecture
\citep{peng2009conditional,zheng2015conditional,chen2015learning}
as well as in deeper energy-based architectures
\citep{belanger2016structured,belanger2017end,belanger2017deep,wang2016proximal}.
We further discuss Structured Prediction Energy Networks (SPENs)
in \cref{sec:bg:spens}.

An ongoing discussion in the community argues whether
adding the dependencies explicitly in an energy-based is useful or not.
Feedforward models have a remarkable representational
capacity that can implicitly learn the dependencies and
relationships from data without needing to impose
additional structure or modeling assumptions and
without making the model more computationally
expensive with an optimization-based inference procedure.
One argument against this viewpoint that supports
energy-based modeling is that explicitly including
modeling information improves the data efficiency
and requires less samples to learn because some
structure and knowledge is already present in the model
and does not have to be learned from scratch.

\subsection{Energy-based Models Subsume Feedforward Models}
\label{sec:bg:ff-energy}
We highlight the power of energy-based modeling for supervised
learning by noting how they subsume deep feedforward models.
Let $\hat y = f_\theta(x)$ be a deep feedforward model.
The energy-based representation of this model is
$E(x,y) = ||y-f_\theta(x)||_2^2$
and inference becomes the convex optimization problem
$\hat y = \argmin_y E(x,y)$, which
has the exact solution $\hat y = f_\theta(x)$.
An energy function that has more structure over the output
space adds representational capacity that a feedforward
model wouldn't be able to capture explicitly.

\subsection{Structured Prediction Energy Networks}
\label{sec:bg:spens}
Structured Prediction Energy Networks (SPENs)
\citep{belanger2016structured,belanger2017end,belanger2017deep}
are a way of bridging the gap between modern deep
learning methods and classical energy-based learning
methods.
SPENs provide a deep structure over input and output spaces
by representing the energy function $E_\theta(x,y)$ with
a standard feed-forward neural network.
This expressive formulation comes at the cost of making
the inference procedure in \cref{eq:bg:inf} difficult
and non-convex.
SPENs typically use an approximate inference procedure
by taking a fixed-number of gradient descent steps
for inference.
For learning, SPENs typically replace the inference with
an \emph{unrolled gradient-based optimizer}
that starts with some prediction $y_0$ and
takes a fixed number of gradient steps to minimize
the energy function
$$y_{i+1} = y_i - \alpha \nabla_y E_\theta(x,y_i).$$
The final iterate as then taken as the prediction
$\hat y \triangleq y_N$.
Gradient-based parameter learning can be done by
differentiating the prediction $\hat y$ with respect
to $\theta$ by unrolling the inference procedure.
Unrolling the inference procedure
can be done in most autodiff frameworks such as
PyTorch \citep{paszke2017automatic}
or TensorFlow \citep{abadi2016tensorflow}.
activation functions with smooth first derivatives
such as the sigmoid or softplus \citep{glorot2011deep}
should be used to avoid discontinuities because
unrolling the inference procedure involves computing
$\nabla_\theta \nabla_y E_\theta(x,y)$.

\section{Modeling with Domain-Specific Knowledge}
\label{sec:bg:dsn}
The role of domain-specific knowledge in the machine learning
and computer vision fields has been an active discussion topic
over the past decade and beyond.
Historically, domain knowledge such as fixed hand-crafted
feature and edge detectors were rigidly part of the
computer vision pipeline and have been overtaken by
learnable convolutional models \citet{lecun1998mnist,krizhevsky2012imagenet}.
To highlight the power of convolutional architectures, they
provide a reasonable prior for vision tasks even without
learning \citep{ulyanov2018deep}.
Machine learning models extend far beyond the reach
of vision tasks and the community has a growing interest
on domain-specific priors rather than just using
fully-connected architectures.
These priors ideally can be integrated as end-to-end
learnable modules into a larger system that are
learned as a whole with gradient-based information.
In contrast to pure fully-connected architectures,
specialized submodules ideally improve the data
efficiency of the model, add interpretability,
and enable grey-box verification.

Recent work has gone far beyond the classic examples of
adding modeling priors by using convolutional or
sequential models.
A full discussion of all of the
recent improvements is beyond the scope of this thesis,
and here we highlight a few key recent
developments.
% \todon{Re-phrase some of these}
\begin{itemize}
\item Differentiable beam search \citep{goyal2018continuous}
  and differentiable dynamic programming \citep{mensch2018differentiable}
\item Differentiable protein simulator \citep{ingraham2018learning}
\item Differentiable particle filters \citep{jonschkowski2018differentiable}
\item Neural ordinary differential equations \citep{chen2018neural}
  and applications to reversible generative models \citep{grathwohl2018ffjord}
\item Relational reasoning on sets, graphs, and trees
  \citep{battaglia2018relational,zaheer2017deep,
    kipf2016semi,gilmer2017neural,santoro2017simple,
    hamilton2017inductive,battaglia2016interaction,
    xu2018powerful,farquhar2017treeqn,shen2018ordered}
\item Geometry-based priors
  \citep{bronstein2017geometric,gulcehre2018hyperbolic,
     monti2017geometric,tang2018ba,li2018smoothing}
\item Memory \citep{sukhbaatar2015end,graves2014neural,graves2016hybrid,xiong2016dynamic,hill2015goldilocks,parisotto2017neural}
\item Attention \citep{bahdanau2014neural,vaswani2017attention,wang2018non}
\item Capsule networks \citep{sabour2017dynamic,hinton2018matrix,xinyi2018capsule}
\item Program synthesis
\citep{reed2015neural,neelakantan2015neural,balog2016deepcoder,devlin2017robustfill,parisotto2016neuro}
\end{itemize}

\section{Optimization-based Modeling}
\label{sec:bg:opt}
Optimization can be used for modeling in machine learning.
Among many other applications, these architectures are well-studied for
generic classification and structured prediction tasks
\citep{goodfellow2013multi,stoyanov2011empirical,brakel2013training,lecun2006tutorial,belanger2016structured,belanger2017end};
in vision for tasks such as denoising
\citep{tappen2007learning,schmidt2014shrinkage}
or edge-aware smoothing \citep{barron2016fast}.
\citet{diamond2017unrolled} presents unrolled optimization
with deep priors.
\citet{metz2016unrolled} uses unrolled optimization within a network to
stabilize the convergence of generative adversarial networks
\citep{goodfellow2014generative}.
Indeed, the general idea of solving restricted classes of
optimization problem using neural networks goes back
many decades \citep{kennedy1988neural, lillo1993solving},
but has seen a number of advances in recent years.
These models are often trained by one of
the following four methods.

\subsection{Explicit Differentiation}
If an analytic solution to the argmin can be found,
such as in an unconstrained quadratic minimization,
the gradients can often also be computed analytically.
This is done in \citet{tappen2007learning,schmidt2014shrinkage}.
We cannot use these methods for
the constrained optimization problems
we consider in this thesis because
there are no known analytic solutions.

\subsection{Unrolled Differentiation}
\label{sec:bg:unroll}
The argmin operation over an unconstrained objective can be approximated
by a first-order gradient-based method and unrolled.
These architectures typically introduce an optimization
procedure such as gradient descent into the inference procedure.
This is done in
\citet{domke2012generic,belanger2017end,metz2016unrolled,goodfellow2013multi,stoyanov2011empirical,brakel2013training,finn2017model}.
The optimization procedure is unrolled automatically or manually
\citep{domke2012generic} to obtain derivatives during training that incorporate
the effects of these in-the-loop optimization procedures.

Given an unconstrained optimization problem with a
parameterized objective
$$\argmin_x f_\theta(x),$$
gradient descent starts at an initial value $x_0$ and
takes steps $$x_{i+1} = x_i - \alpha \nabla_x f_\theta(x).$$
For learning, the final iterate of this procedure $x_N$ can
be taken as the output and
$\partial x_N/\partial \theta$ can
be computed with automatic differentiation.

In all of these cases, the optimization problem is unconstrained
and unrolling gradient descent is often easy to do.
When constraints are added to the optimization problem, iterative
algorithms often use a projection operator that may be difficult
to unroll through and storing all of the intermediate iterates
may become infeasible.

\subsection{Implicit argmin differentiation}
\label{sec:bg:argmin-diff}
Most closely related to this thesis work, there have been
several applications of the implicit function theorem
to differentiating through constrained convex argmin
operations.
These methods typically parameterize an optimization problem's
objective or constraints and then applies the
\emph{implicit function theorem} (\cref{theorem:implicit})
to optimality conditions
of the optimization problem that implicitly define
the solution, such as the \emph{KKT conditions}
\citep[Section~5.5.3]{boyd2004convex}.
We will first review the implicit function theorem
and KKT conditions and then discuss related work
in this space.

\emph{Implicit function} analysis
\citep{dontchev2009implicit}
typically focuses on solving an equation $f(p,x)=0$ for $x$ as
a function $s$ of $p$, \ie $x=s(p)$.
\emph{Implicit differentiation} considers how to
differentiate the solution mapping with respect
to the parameters, \ie $\nabla_p s(p)$.
The \emph{implicit function theorem} used in standard
calculus textbooks can be traced back to the
lecture notes from 1877-1878 of \citet{dini1877analisi}
and is presented in
\citet[Theorem~1.B.1]{dontchev2009implicit} as follows.

\begin{theorem}[Implicit function theorem]
  \label{theorem:implicit}
  Let $f: \RR^d \times \RR^n \rightarrow \RR^n$ be continuously
  differentiable in a neighborhood of
  $(\bar p, \bar x)$ and such that $f(\bar p, \bar x)=0$,
  and let the partial Jacobian of $f$ with respect to
  $x$ at $(\bar p, \bar x)$, namely $\nabla_x f(\bar p, \bar x)$,
  be nonsingular. Then the solution mapping
  $S(p) = \menge{x\in \RR^n}{f(p,x)=0}$ has a single-valued
  localization $s$ around $\bar p$ for $\bar x$ which
  is continuously differentiable in a neighborhood $Q$
  of $\bar p$ with Jacobian satisfying
  $\nabla s(p) = -\nabla_x f(p, s(p))^{-1} \nabla_p f(p, s(p))$
  for every $p\in Q$.
\end{theorem}

In addition to the content in this thesis, several other papers
apply the implicit function theorem to differentiate through
the argmin operators.
This approach frequently comes up in bilevel optimization
\citep{gould2016differentiating,kunisch2013bilevel}
and sensitivity analysis
\citep{bertsekas1999nonlinear,fiacco1990sensitivity,bell2008algorithmic,bonnans2013perturbation}.
\citep{barratt2018differentiability} is a note on applying the
implicit function theorem to the KKT conditions of convex
optimization problems and highlights assumptions behind the
derivative being well-defined.
\citet{gould2016differentiating} describes general techniques for
differentiation through optimization problems,
but only describe the case of exact equality constraints rather than
both equality and inequality constraints
(they add inequality constraints via a barrier function).
\citet{johnson2016composing} performs implicit differentiation on
(multi-)convex objectives with coordinate subspace constraints.
The older work of \citet{mairal2012task} considers argmin
differentiation for a LASSO problem, derives specific rules for this case, and
presents an efficient algorithm based upon our ability to solve
the LASSO problem efficiently.
\citet{jordan2015convex} studies convex optimization over probability
measures and implicit differentiation in this context.
\citet{bell2008algorithmic} adapts automatic differentiation
to obtain derivatives of implicitly defined functions.

\subsection{An optimization view of the ReLU, sigmoid, and softmax}
\label{sec:bg:existing}
In this section we note how the commonly used ReLU, sigmoid, and softmax
functions can be interpreted as explicit closed-form solutions
to constrained convex optimization (argmin) problems.
\citet{bibi2018deep} presents another view that interprets other
layers as proximal operators and stochastic solvers.
We use these as examples to further highlight the power of
optimization-based inference, not to provide a new analysis
of these layers. The main focus of this thesis is \emph{not}
on learning and re-discovering existing activation functions.
In this thesis, we rather propose new optimization-based inference
layers that do \emph{not} have explicit closed-form solutions
like these examples and show that they can still be efficiently
turned into differentiable building blocks for end-to-end architectures.

\begin{theorem}
  The ReLU, defined by $f(x) = \max\{0, x\}$,
  can be interpreted as projecting a point $x\in\RR^n$ onto
  the non-negative orthant as
  \begin{equation}
    f(x) = \argmin_y \;\; \frac{1}{2}||x-y||_2^2 \;\; \st \;\; y\geq 0.
    \label{eq:relu-proj}
  \end{equation}
\end{theorem}

\begin{proof}
  The usual solution can be obtained by looking at
  the KKT conditions of \cref{eq:relu-proj}.
  Introducing a dual variable $\lambda\geq 0$ for the inequality
  constraint, the Lagrangian of \cref{eq:relu-proj} is
  \begin{equation}
    L(y, \lambda)=\frac{1}{2}||x-y||_2^2 - \lambda^\top y.
  \end{equation}
  The stationarity condition
  $\nabla_y L(y^\star, \lambda^\star) = 0$
  gives a way of expressing the primal optimal
  variable $y^\star$ in terms of the dual optimal
  variable $\lambda^\star$ as $y^\star=x+\lambda^\star$.
  Complementary slackness $\lambda^\star_i(x_i+\lambda^\star_i)=0$
  shows that $\lambda^\star_i\in\{0, -x_i\}$.
  Consider two cases:
  \begin{itemize}
  \item \textbf{Case 1:} $x_i\geq 0$.
    Then $\lambda^\star_i$ must be 0
    since we require $\lambda^\star\geq 0$.
    Thus $y^\star_i=x_i+\lambda^\star_i=x_i$.
  \item \textbf{Case 2:} $x_i< 0$.
    Then $\lambda^\star_i$ must be $-x_i$
    since we require $y\geq 0$.
    Thus $y^\star_i=x_i+\lambda^\star_i=0$.
  \end{itemize}
  Combining these cases gives the usual solution of
  $y^\star=\max\{0, x\}$.
\end{proof}

\newpage
\begin{theorem}
  The sigmoid or logistic function, defined by $f(x) = (1+e^{-x})^{-1}$,
  can be interpreted as projecting a point $x\in\RR^n$ onto
  the interior of the unit hypercube as
  \begin{equation}
    f(x) = \argmin_{0<y<1} \;\; -x^\top y -H_b(y),
    \label{eq:sigmoid-proj}
  \end{equation}
  where $H_b(y) = - \left(\sum_i y_i\log y_i + (1-y_i)\log (1-y_i)\right)$ is the
  binary entropy function.
\end{theorem}

\begin{proof}
  The usual solution can be obtained by looking at
  the first-order optimality condition of
  \cref{eq:sigmoid-proj}.
  The domain of the binary entropy function $H_b$ restricts
  us to $0<y<1$ without needing to explicitly represent this
  as a constraint in the optimization problem.
  Let $g(y; x) = -x^\top y -H_b(y)$ be the objective.
  The first-order optimality condition $\nabla_y g(y^\star; x) = 0$
  gives us $-x_i + \log y_i^\star - \log (1-y_i^\star) = 0$
  and thus $y^\star = (1+e^{-x})^{-1}$.
\end{proof}

\begin{theorem}
  The softmax, defined by $f(x)_j = e^{x_j} / \sum_i e^{x_i}$,
  can be interpreted as projecting a point $x\in\RR^n$ onto
  the interior of the $(n-1)$-simplex
  $$\Delta_{n-1}=\{p\in\RR^n\; \vert\; 1^\top p = 1 \; \; {\rm and} \;\; p \geq 0 \}$$
  as
  \begin{equation}
    f(x) = \argmin_{0<y<1} \;\; -x^\top y - H(y) \;\; \st\;\; 1^\top y = 1
    \label{eq:simplex-proj}
  \end{equation}
  where $H(y) = -\sum_i y_i \log y_i$ is the entropy function.
\end{theorem}

\begin{proof}
  The usual solution can be obtained by looking at
  the KKT conditions of \cref{eq:simplex-proj}.
  Introducing a scalar-valued dual variable $\nu$ for the
  equality constraint, the Lagrangian is
  \begin{equation}
    L(y, \nu) = -x^\top y - H(y) + \nu(1^\top y - 1)
  \end{equation}
  The stationarity condition
  $\nabla_y L(y^\star, \nu^\star) = 0$
  gives a way of expressing the primal optimal
  variable $y^\star$ in terms of the dual optimal
  variable $\nu^\star$ as
  \begin{equation}
    \label{eq:simplex-proj-pd}
    y^\star_j=\exp\{x_j-1-\nu^\star\}.
  \end{equation}
  Putting this back into the equality constraint
  $1^\top y^\star = 1$ gives us
  $\sum_i \exp\{x_i-1-\nu^\star\} = 1$ and thus
  $\nu^\star = \log\sum_i\exp\{x_i-1\}$.
  Substituting this back into \cref{eq:simplex-proj-pd}
  gives us the usual definition of
  $y_j = e^{x_j} / \sum_i e^{x_i}$.
\end{proof}

\begin{corollary}
A temperature-scaled softmax scales the entropy term in the objective
and the sparsemax \citep{martins2016softmax} replaces the
objective's entropy penalty with a ridge section.
\end{corollary}

\newpage
\section{Reinforcement Learning and Control}
\label{sec:bg:rl}

The fields of reinforcement learning (RL) and optimal control
typically involve creating agents that act optimally
in an environment.
These environments can typically be represented
as a Markov decision process (MDP) with a continuous
or discrete state space and a continuous or discrete action space.
The environment often has some oracle-given
reward associated with each state and the goal of RL
and control is to find a policy that maximizes the
cumulative reward achieved.

Using the notation from \citep{levine2017introduction},
\emph{policy search} methods learn a policy $\pi_\theta(u_t|x_t)$
parameterized by $\theta$ that predicts a distribution over next
action to take given the current state $x_t$.
The goal of policy search is to find a policy that maximizes the
expected return
\begin{equation}
  \argmax_\theta\; \E_{\tau\sim p_\theta(\tau)} \left[ \sum_t \gamma^t r(x_t, u_t) \right],
\end{equation}
where $p_\theta(\tau)=p(x_1)\prod\pi_\theta(u_t|x_t)p(x_{t+1}|x_t,u_t)$
is the distribution over trajectories,
$\gamma\in(0,1]$ is a discount factor,
$r(x_t, u_t)$ is the state-action reward at time $t$,
and $p(x_{t+1}|x_t,u_t)$ is the state-transition probability.
In many scenarios, the reward $r$ is assumed to be
a black-box function that derivative information cannot
be obtained from.
\emph{Model-free} techniques for policy search typically do
not attempt to model the state-transition probability
while \emph{model-based} and \emph{control} approaches do.

\emph{Control approaches} typically provide a policy
by planning based on known state transitions.
For example, in continuous state-action spaces with deterministic
state transitions, the finite-horizon model predictive control problem is
\begin{equation}
    \argmin_{x_{1:T} \in \mathcal{X},u_{1:T}\in \mathcal{U}} \;\; \sum_{t=1}^T  C_t(x_t, u_t)
    \;\; \subjectto \;\; x_{t+1} = f(x_t, u_t), \;\; x_1 = \xinit,
\end{equation}
where $\xinit$ is the current system state,
the cost $C_t$ is typically hand-engineered and differentiable,
and $x_{t+1}=f(x_t, u_t)$ is the deterministic
next-state transition, \ie~the point-mass given by $p(x_{t+1}|x_t,u_t)$.
While this thesis focuses on the continuous and deterministic setting,
control approaches can also be applied in discrete
and stochastic settings.

\textbf{Pure model-free techniques for policy search} have
demonstrated promising results in many domains by learning
\emph{reactive polices} which directly map observations to actions
\citep{mnih2013playing,oh2016minecraft,gu2016continuous,lillicrap2015continuous,
schulman2015trust,schulman2016trpogae,gu2017qprop}.
Despite their success, model-free methods have many drawbacks and limitations,
including a lack of interpretability, poor generalization, and a
high sample complexity.
\textbf{Model-based methods} are known to be more sample-efficient
than their model-free
counterparts.
These methods generally rely on learning a dynamics model directly from
interactions with the real system and then integrate the learned model into the
control policy
\citep{schneider1997exploiting,abbeel2006using,deisenroth2011pilco,heess2015learning,
boedecker2014sparsegps}.
More recent approaches use a deep network to learn low-dimensional latent state
representations and associated dynamics models in this learned representation.
They then apply standard trajectory optimization methods
on these learned embeddings
\citep{lenz2015deepmpc, watter2015embed, levine2016end}.
However, these methods still require a manually specified and hand-tuned
cost function, which can become even more difficult in a latent representation.
Moreover, there is no guarantee that the learned dynamics model
can accurately capture portions of the state space relevant for the task at hand.

To leverage the benefits of both approaches, there has been significant
interest in \textbf{combining the model-based and model-free paradigms.}
In particular, much attention has been dedicated to utilizing
model-based priors to accelerate the model-free learning process.
For instance, synthetic training data can be generated by model-based control
algorithms to guide the policy search or prime a
model-free policy
\citep{sutton1990integrated,theodorou2010generalized,levine2014learning,
gu2016continuous,venkatraman2016improved,levine2016end,chebotar2017combining,
nagabandi2017mbmf, liting2017driving}.
\citep{bansal2017mbmf} learns a controller and then distills it to
a neural network policy which is then fine-tuned with model-free
policy learning.
However, this line of work usually keeps the model separate from the
learned policy.

Alternatively, the policy can include an \textbf{explicit planning module}
which \emph{leverages learned models} of the system or environment,
both of which are learned through model-free techniques.
For example, the classic Dyna-Q algorithm
\citep{sutton1990integrated} simultaneously learns a model of
the environment and uses it to plan.
More recent work has explored incorporating such structure into deep
networks and learning the policies in an end-to-end fashion.
\citet{tamar2016value} uses a recurrent network to predict the value function by
approximating the value iteration algorithm with convolutional layers.
\citet{karkus2017qmdp} connects a dynamics model to a planning
algorithm and formulates the policy as a structured recurrent network.
\citet{silver2016predictron} and \citet{oh2017value} perform multiple rollouts
using an abstract dynamics model to predict the value function.
A similar approach is taken by \citet{weber2017imagination} but directly
predicts the next action and reward from rollouts of an explicit environment model.
\citet{farquhar2017treeqn} extends model-free approaches, such as
DQN \citep{mnih2015human} and A3C \citep{mnih2016asynchronous}, by planning
with a tree-structured neural network to predict the cost-to-go.
While these approaches have demonstrated impressive results in
discrete state and action spaces, they are not applicable to
continuous control problems.

To tackle continuous state and action spaces, \citet{pascanu2017learning}
propose a neural architecture which uses an abstract environmental
model to plan and is trained directly from an external task loss.
\citet{pong2018temporal} learn goal-conditioned value functions and use them
to plan single or multiple steps of actions in an MPC fashion.
Similarly, \citet{pathak2018zero} train a goal-conditioned policy to perform
rollouts in an abstract feature space but ground the policy with a loss term
which corresponds to true dynamics data.
The aforementioned approaches can be interpreted as a distilled optimal controller
which does not separate components for the cost and dynamics.
Taking this analogy further, another strategy is to differentiate through an
optimal control algorithm itself.
\citet{okada2017path} and \citet{pereira2018pinets} present a way
to differentiate through path integral optimal control
\citep{williams2016aggressive,williams2017model}
and learn a planning policy end-to-end.
\citet{srinivas2018universal} shows how to embed
differentiable planning (unrolled gradient descent over actions) within
a goal-directed policy.
In a similar vein, \citet{tamar2017learning} differentiates
through an iterative LQR (iLQR) solver
\citep{li2004ilqr,xie2017ddp,tassa2014control}
to learn a cost-shaping term offline.
This shaping term enables a shorter horizon controller to approximate the
behavior of a solver with a longer horizon to save computation during runtime.

%%% Local Variables:
%%% coding: utf-8
%%% mode: latex
%%% TeX-engine: xetex
%%% TeX-master: "../thesis"
%%% End:


\part{Foundations}
\include{sections/optnet}
\include{sections/icnn}

\part{Extensions and Applications}
\include{sections/empc}
\include{sections/lml}
\graphicspath{{cvxpyth/}}

\chapter{Differentiable \cvxpy Optimization Layers}
\label{sec:cvxpyth}

In this chapter, we show how to turn the \cvxpy
modeling language \citep{diamond2016cvxpy} into
a differentiable optimization layer and
implement our method in PyTorch \citep{paszke2017automatic}.
This allows users to express convex optimization layers in
the intuitive \cvxpy modeling language without needing
to manually implement the backward pass.

\section{Introduction}
This thesis has presented differentiable optimization layers
as a powerful class of operations for end-to-end learning
that allow more specialized domain knowledge to be integrated
into the modeling pipeline in a differentiable way.
Convex optimization layers can be represented as
\begin{equation}
  \label{eq:optimization-layer}
  z_{i+1} = \argmin_z f_\theta(z, z_i)\;
  {\rm s.t.}\;z\in \CC_\theta(z_i)
\end{equation}
where $z_i$ is the previous layer,
$f$ is a convex objective function parameterized
by $\theta$, and $\CC$ is a convex constraint set.
From the perspective of end-to-end learning,
convex optimization layers can be seen
as a module that outputs $z_{i+1}$ and has parameters
$\theta$ that can be learned with gradient descent.
We note that the convex case captures many
of the applications above, and can be used as
a building block for differentiable non-convex
optimization problems.

Implementing optimization layers can be non-trivial as explicit
closed-form solutions typically do not exist.
The forward pass needs to call into an optimization
problem solver and the backward pass typically
\emph{cannot} leverage automatic differentiation.
The backwards pass is usually implemented by implicitly
differentiating the KKT conditions of the optimization problem
as done in bilevel optimization
\citep{gould2016differentiating,kunisch2013bilevel},
sensitivity analysis
\citep{bertsekas1999nonlinear,fiacco1990sensitivity,bonnans2013perturbation},
and in our OptNet approach \cref{sec:optnet}.
Thus to implement an optimization layer, users
have to manually implement the backwards pass,
which is cumbersome and error-prone,
or use an existing optimization problem layer such as
the differentiable QP layer from \cref{sec:optnet},
which is not capable of exploiting problem-specific
structures, uses dense operations, and requires
the user to manually transform their problem into
a standard form.

We make \cvxpy differentiable with respect to the
\texttt{Parameter} objects provided to the optimization problem
by making internal \cvxpy components differentiable.
This involves differentiating the reduction from the \cvxpy
language to the problem data of a cone program in standard form
and then differentiating through the cone program.
We show how to differentiate through cone programs by
implicitly differentiating the residual map from
\citet{busseti2018solution},
which is of standalone interest as
this shows how to differentiate through optimization
problems with non-polytope constraints.

\section{Background}
\subsection{The \cvxpy modeling language}
\label{sec:bg:cvxpy}
\cvxpy \citep{diamond2016cvxpy}
is a domain-specific modeling language
based on disciplined convex programming
\citep{grant2006disciplined}
that allows users to express optimization problems
in a more natural way than the standard form
required my most optimization problem solvers.
\cvxpy works by transforming the optimization
problem from their domain-specific language to a
standard (or canonical) form that is passed into
a solver. This inner canonicalized problem is
then solved and the results are returned to
the user. In this chapter, we focus on the
canonicalization to a cone program, which is
one of the most commonly used modes as most convex
optimization problems can be expressed as a cone program,
although we note that our method can be
applied to other \cvxpy solvers.
\cref{fig:overview} overviews the relevant
\cvxpy components.

\subsection{Cone Preliminaries}
A set $\mathcal{K}$ is a \emph{cone}
if for all $x\in\mathcal{K}$ and $t>0$,
$tx\in\mathcal{K}$.
The \emph{dual cone} of a cone $\mathcal{K}$ is
$$\mathcal{K}^* =\menge{y}{\inf_{x\in \mathcal{K}} y^\top x\ge 0}.$$
Commonly used cones include the
nonnegative orthant $\menge{x}{x\geq 0}$,
second-order cone $\menge{(x,t)\in\RR^n_+}{t\geq ||x||_2}$,
positive semidefinite cone $\{X=X^\top \succeq 0\}$,
and
exponential cone
\begin{equation}
  \menge{(x,y,z)}{y>0, ye^{x/y} \leq z} \cup
  \menge{(x,0,z)}{x\leq 0, z\geq 0}
\end{equation}
We can also create a cone from the Cartesian
products of simpler cones as
$\mathcal{K}=\mathcal{K}_1\times \ldots \times\mathcal{K}_p$.

\subsection{Cone Programming}
Most convex optimization problems can be represented and
efficiently solved as a cone program that uses
the nonnegative orthant, second-order cone,
positive semidefinite cone, and exponential cones.
This applicability makes them a commonly used internal
solver for \cvxpy, which implements many of the
well-known transformations from problems to their
conic form.
In the following we state properties of cone programs
and useful definitions for this chapter.
More details about cone programming can be found in
\citet{boyd2004convex,ben2001lectures,busseti2018solution,odonoghue2016conic,lobo1998applications,alizadeh2003second}.

In their primal (P) and dual (D) forms,
cone programs can be represented as \\
\begin{minipage}{0.45\textwidth}
  \begin{equation*}
    % \tag{P}
    \begin{array}{lll}
      \text{(P)}\;\; \xstar, \sstar =
      &\argmin_{x,s} &c^\top x\\
      &\subjectto &  Ax+s=b\\
      &  &s\in  \mathcal{K}
    \end{array}
  \end{equation*}
  \vspace{3mm}
\end{minipage}
\hfill
\begin{minipage}{0.5\textwidth}
  \begin{equation}
    \label{eq:cp}
    % \tag{D}
    \begin{array}{lll}
      \text{(D)}\;\; \ystar =
      & \argmax_y & b^\top y\\
      &\subjectto & A^\top y+c=0\\
      &&y\in  \mathcal{K}^*
    \end{array}
  \end{equation}
  \vspace{3mm}
\end{minipage}
where $x\in\R^n$ is the \emph{primal variable},
$s\in\R^m$ is the \emph{primal slack variable},
$y\in\R^m$ is the \emph{dual variable}.
and $\mathcal{K}$ is a nonempty, closed, convex cone
with dual cone $\mathcal{K}^*$.

\textbf{The KKT optimality conditions.}
The Karush--Kuhn--Tucker (KKT) conditions for the
cone program in \cref{eq:cp} provide
necessary and sufficient conditions for optimality
and are defined by
\begin{equation}
\label{eq:cp-kkt}
Ax +s =b, \quad
A^\top y + c = 0, \quad
s \in \mathcal{K}, \quad
y \in \mathcal{K}^*, \quad
s^\top y = 0.
\end{equation}
The complimentary slackness condition $s^\top y = 0$ can
alternatively be captured with a condition that
makes the duality gap zero $c^\top x + b^\top y = 0$.

\textbf{The homogenous self-dual embedding.}
\citet{ye1994nl} converts the primal and cone dual programs
in \cref{eq:cp} into a single feasibility problem called
the homogenous self-dual embedding, which is defined by
\begin{equation}
\label{e:hsde:1}
Qu = v, \quad u \in \mathbfcal{K},
\quad v \in \mathbfcal{K}^*, \quad u_{m+n+1} +
 v_{m+n+1} >0,
\end{equation}
where
\[
\mathbfcal{K} = \RR^n \times \mathcal{K}^*\times \RR_+, \quad
\mathbfcal{K}^* = \{0\}^n\times \mathcal{K}\times \RR_+,
\]
and $Q$ is the skew-symmetric matrix
\[
	Q = \begin{bmatrix}
		0 & A{^\top} & c\\
		-A & 0 & b \\
		-c{^\top} & -b{^\top} & 0
	\end{bmatrix}.
\]
A solution to this embedding problem $(\ustar, \vstar)$
can be used to determine the solution of a conic
problem, or to certify the infeasibilty of the
problem if a solution doesn't exist.
If a solution exists, then
$\ustar=(\xstar/\tau,\ystar/\tau,\tau)$
and
$\vstar=(0, \sstar/\tau, 0)$.

\textbf{The conic complementarity set.}
The \emph{conic complementarity set} is defined by
\begin{equation}
\label{eq:con:com:set}
\mathcal{C} = \menge{(u,v) \in
  \mathbfcal{K} \times \mathbfcal{K}^* }{u^\top v = 0}.
\end{equation}
We denote
the Euclidean projection onto
$\mathbfcal{K}$ with $\Pi$
and
the Euclidean projection onto
$-\mathbfcal{K}^*$ with $\Pi^*$.
\citet{moreau1961decomposition} shows that
$\Pi^*=I-\Pi$.

\textbf{Minty's parameterization of the complementarity set.}
Minty's parameterization of $\mathcal{C}$
$M: \RR^{m+n+1} \to \mathcal{C}$ as
$M(z) = (\Pi z, -\Pi^* z)$.
This parameterization is invertible with
$M^{-1}(u,v) = u-v$.
See
\citet[Corollary~31.5.1]{rockafellar1970convex}
and \citet[Remark~23.23(i)]{bauschke2017convex}
for more details.
The homogeneous self-dual embedding can be expressed
using Minty's parameterization as
$-\Pi^* z=Q\Pi z$ where $z_{m+n+1} \neq 0$.

\textbf{The residual map of Minty's parameterization.}
\label{sec-residual-map}
\citet{busseti2018solution} defines the
\emph{residual map} of Minty's parameterization
$\Res: \RR^{m+n+1}\to \RR^{m+n+1}$
as
\begin{equation}
\label{eq-residual}
\Res(z) = Q\Pi z+\Pi^*z=((Q-I)\Pi+I)z.
\end{equation}
and shows how to compute the derivative of it
when $\Pi$ is differentiable at $z$ as
\begin{equation}
\label{eq:res:der}
\DD_z\Res(z) =(Q-I)\DD_z\Pi(z) +I,
\end{equation}
where $z\in \RR^{m+n+1}$.
The cone projection differentiation $\DD_z \Pi(z)$
can be computed as described in
\citet{ali2017semismooth}.

\textbf{The Splitting Conic Solver (SCS).}
SCS \citep{odonoghue2016conic} is an efficient way of solving
general cone programs by using the alternating
direction method of multipliers (ADMM)
\citep{boyd2011distributed} and is a commonly used
solver with \cvxpy.
In the simplified form, each iteration
of SCS consists of the following three steps:
\begin{equation}
\begin{array}{rcl}
\tilde u^{k+1} &=& (I + Q)^{-1} (u^k + v^k )\\[1ex]
u^{k+1} &=& \Pi\left(\tilde u^{k+1} - v^k\right) \\[1ex]
v^{k+1} &=&  v^k - \tilde u^{k+1} + u^{k+1}.
\end{array}
\label{eq:scs}
\end{equation}

The first step projects onto an affine subspace,
the second projects onto the cone
and the last updates the dual variable.
In this paper we will mostly focus on solving the
affine subspace projection step. \citet[Section 4.1]{odonoghue2016conic}
shows that the affine subspace projection can be
reduced to solving linear systems of the form
\begin{equation}
\label{eq:scs-linsys}
\begin{bmatrix}
I & -A^\top \\
-A & -I  \\
\end{bmatrix}
\begin{bmatrix} z_x \\ -z_y \end{bmatrix}
=
\begin{bmatrix} w_x \\ w_y \end{bmatrix},
\end{equation}
which can be re-written as
\begin{equation}
  \label{eq:scs-linsys-elim}
  z_x = (I + A^\top A)^{-1}(w_x - A^\top w_y), \quad
  z_y = w_y + A z_x.
\end{equation}

\section{Differentiating \cvxpy and Cone Programs}
\label{sec:cvxpyth:diff-cp}

\begin{figure}[t]
  \centering
  \includegraphics[width=0.9\textwidth]{overview.pdf}
  \caption{
    Summary of our differentiable \cvxpy layer that allows
    users to easily turn most convex optimization problems into
    layers for end-to-end machine learning.
  }
  \label{fig:overview}
\end{figure}

We have created a differentiable \cvxpy layer
by making the relevant components differentiable,
which we visually show in \cref{fig:overview}.
We make the transformation from the problem data in the original
form to the problem data of the canonicalized cone problem
differentiable by replacing the numpy operations for this component
with PyTorch \citep{paszke2017automatic} operations.
We then pass this data into a differentiable cone program
solver, which we show how to create in \cref{sec:diff-cp}
by implicitly differentiating the residual map of
Minty's parameterization for the backward pass.
The solution of this cone program can then be mapped
back up to the original problem space in a differentiable
way and returned.

\subsection{Differentiating Cone Programs}
\label{sec:diff-cp}
We consider the argmin of the primal cone program
in \cref{eq:cp} as a function that maps from the
problem data $\theta = \{c, A, b\}$ to the solution $\xstar$.
The approach from \cref{sec:optnet}
that differentiates through convex quadratic programs by
implicitly differentiating the KKT conditions
is difficult to use for cone programs.
This is because the cone constraints in the KKT conditions
in \cref{eq:cp-kkt} make it difficult to form a
set of implicit functions.
Instead of implicitly differentiating the KKT conditions,
we show how to similarly apply implicit
differentiation to the residual map of
Minty's parameterization shown in \citet{busseti2018solution}
to compute the derivative
$\partial \xstar/\partial \theta$.
Furthermore for backpropagation, the full Jacobian is
expensive and unnecessary to form and we show how to
efficiently compute $\partial \ell/\partial \theta$
given $\partial \ell/\partial \xstar$.

\subsubsection{Implicit differentiation of the residual map}
We assume that we have solved the forward pass of the
cone program in \cref{eq:cp} and have a solution
$\xstar, \sstar, \ystar$.
We now show how to compute
$\partial \xstar/\partial \theta$.
This derivation was concurrently considered and done by
\citet{agrawal2019differentiating}.

We construct $\ustar=(\xstar, \ystar, 1)$,
and $\vstar=(0,\sstar, 0)$, and $\zstar=\ustar-\vstar$.
The residual map of Minty's parameterization is zero,
$R(\zstar)=0$, and forms a set of
implicit equations that describe the solution mapping.
Implicit differentiation can be done as described in
\citet{dontchev2009implicit} with
\begin{equation}
  \label{eq:residual-implicit-diff}
  \DD_\theta \zstar =
    -\left(\DD_z \Res(\zstar)\right)^{-1}
    \DD_\theta \Res(\zstar).
\end{equation}
$\left(\DD_z \Res(\zstar)\right)^{-1}$ can be computed
as described in \citep{busseti2018solution}
and $\DD_\theta \Res(\zstar)$ can be analytically computed.
We consider the scaling factor $\tau=z_{m+n+1}=1$ to be
a constant because a solution to the cone program exists.
Finally, applying the chain rule to $\ustar=\Pi \zstar$
gives
\begin{equation}
  \DD_\theta \ustar = (\DD_z \Pi z) \DD_\theta \zstar.
\end{equation}
We note that implicitly differentiating the residual
map captures implicit differentiation of the KKT conditions
as a special case for simple cones such as the zero cone
and non-negative orthant.

The linear system in \cref{eq:residual-implicit-diff}
can be expensive to solve.
In special cases such as quadratic programs and LQR problems
that we discussed in \cref{sec:optnet} and
\cref{sec:empc}, respectively, this system can be interpreted
as the solution to another convex optimization problem
and efficiently solved with a method similar to the
forward pass.
This connection is made by interpreting the linear system
solve as a KKT system solve that represents another
optimization problem.
However for general cone programs it is more difficult
to interpret this linear system as a KKT system
because of the cone projections and therefore it
is more difficult to interpret this linear system
solve as an optimization problem.

\section{Implementation}
\subsection{Forward Pass: Efficiently solving batches of
  cone programs with SCS and PyTorch}
\label{sec:cp:efficient}

Na\"ively implemented optimization layers can become
computational bottlenecks when used in a machine learning
pipeline that requires efficiently processing
minibatches of data.
This is because most other parts of the modeling pipeline
involve operations such as linear maps, convolutions,
and activation functions that can quickly be executed
on the GPU to exploit data parallelism across the minibatch.
Most off-the-shelf optimization problem solvers are designed for
the setting of solving a single problem at a time and are not easily
able to be plugged into the batched setting required
when using optimization layers.

To overcome the computational challenges of solving batches
of cone programs concurrently, we have created a batched
PyTorch and potentially GPU-backed backend for the
Splitting Conic Solver (SCS) \citep{odonoghue2016conic}.
The bottleneck of the SCS iterates in \cref{eq:scs}
is typically in the subspace projection part that solves
linear systems of the form
\begin{equation}
  \tilde u^{k+1} = (I + Q)^{-1} (u^k + v^k )
\end{equation}
We have added a new linear system solver backend to the
official SCS C implementation that calls back
up into Python to solve this linear system.

Our cone program layer implementation offers the following
modes for solving a batch of $N$ cone programs
represented in standard form as in \cref{eq:cp} with
as $\theta_i=\{A_i, b_i, c_i\}$
for $i\in\{1, \ldots, N\}$ with SCS.
As common in practice, we assume that the cone programs
have the structure and use the same cones but have
different problem data $\theta_i$.
We empirically compare these modes in \cref{sec:eval}.

\paragraph{Vanilla SCS, serialized.}
This is a baseline mode that is the easiest to implement and sequentially
iterates through the problems $\theta_i$.
This lets us use the vanilla SCS sparse direct and indirect
linear system solvers on the CPU and CUDA, but does
not take advantage of data parallelism.

\paragraph{Vanilla SCS, batched.}
This is another baseline mode that comes from observing that a
batch of cone programs can be represented as
a single cone program in standard form as in \cref{eq:cp} with
variables $x=[x_1^\top, \ldots, x_N^\top]^\top$
and data $A=\mathrm{diag}(A_1, \ldots, A_N)$,
$b=[b_1^\top, \ldots, b_N^\top]^\top$,
and $c=[c_1^\top, \ldots, c_N^\top]^\top$.
This exploits the knowledge that all of the cone programs
can be solved concurrently. The bottleneck of this
mode is still typically in the linear system solve
portion of SCS, which happens using sparse operations
on the CPU or GPU.

\paragraph{SCS+PyTorch, batched.}
In this mode we represent the batch of cone programs as a single
batched cone program use SCS will callbacks up into Python so
that we can use PyTorch to efficiently solve the linear system.
This allows us to keep the $A$ data in PyTorch and potentially
on the GPU without converting/transferring it and passing it
into the SCS.
Specifically we use dense operations and have implemented
direct and indirect methods to solve \cref{eq:scs-linsys-elim}
in PyTorch and then pass the result back down into SCS for the
rest of the operations.
Our direct method uses PyTorch's batched LU factorizations and
solves and our indirect method uses a batched conjugate
gradient (CG) implementation.
These custom linear system solvers are able to explicitly take
advantage of the independence present in the linear systems
that the sparse linear system solvers may not recognize automatically,
and the dense solvers are also useful for dense cone programs,
which come up in the context of differentiable optimization
layers when large portions of the constraints are being learned.

\subsection{Backward pass: Efficiently solving the linear system}
When using cone programs as layers in end-to-end learning systems
with some scalar-valued loss function $\ell$,
the full Jacobian $\DD_\theta \xstar$ is expensive
and unnecessary to form and requires solving
$|\theta|$ linear systems.
The Jacobian is only used when applying the chain rule
to obtain $\DD_\theta \ell = (\DD_\xstar\ell) \DD_\theta \xstar$.
We can directly compute $\DD_\theta \ell$ without computing
the intermediate Jacobian by solving a single linear system.
Following the method of \cref{sec:optnet}, we set up the system
\begin{equation}
  \label{eq:cvxpyth:diff_efficient}
  \DD_z \Res(\zstar)
\begin{bmatrix}
  d_{z_1} \\
  d_{z_2} \\
  0 \\
\end{bmatrix} = \\
-
\begin{bmatrix}
  \nabla_\xstar \ell \\
  0 \\
  0 \\
\end{bmatrix}.
\end{equation}
Applying the chain rule to $\ustar=\Pi \zstar$ gives
$d_x = d_{z_1}$ and
$d_y = (\DD_z \Pi z) d_{z_2}$.
We then compute the relevant backpropagation derivatives as
\begin{equation}
  \nabla_c \ell = d_x
  \hspace{20mm}
  \nabla_A \ell = d_y \otimes \xstar + \ystar \otimes d_x
  \hspace{20mm}
  \nabla_b \ell = -d_y
\end{equation}

Solving \cref{eq:cvxpyth:diff_efficient} is still challenging
to implement in practice as $\DD_z \Res(\zstar)$ can be large
and sparse and doesn't have obviously exploitable properties such
as symmetry or anti-symmetry.
In addition to directly solving this linear system, we
also explore the use of LSQR \citep{paige1982lsqr} as an
iterative indirect method of solving this system
in \cref{sec:cvxpyth:bw_prof}.
Our LSQR implementation uses the implementation from
\citet{ali2017semismooth} to compute $\DD_z \Pi(z)$
in the form of an abstract linear operator so the
full matrix does not need to be explicitly formed.

\newpage
\section{Examples}
\label{sec:cvxpyth:examples}

This section provides example use cases of our \cvxpy
optimization layer. All of these use the preamble

\begin{lstlisting}
import cvxpy as cp
from cvxpyth import CvxpyLayer
\end{lstlisting}

\subsection{The ReLU, sigmoid, and softmax}
We will start with basic examples and revisit the optimization
views of the ReLU, sigmoid, and softmax from \cref{sec:bg:existing}.
These can be implemented with our \cvxpy layer
in a few lines of code.

\textbf{The ReLU.}
Recall from \cref{eq:relu-proj} that the optimization view is
\begin{equation*}
  f(x) = \argmin_y \;\; \frac{1}{2}||x-y||_2^2 \;\; \st \;\; y\geq 0.
\end{equation*}
We can implement this layer with:
\begin{lstlisting}
x = cp.Parameter(n)
y = cp.Variable(n)
obj = cp.Minimize(cp.sum_squares(y-x))
cons = [y >= 0]
prob = cp.Problem(obj, cons)
layer = CvxpyLayer(prob, params=[x], out_vars=[y])
\end{lstlisting}

This layer can be used and differentiated through
just as any other PyTorch layer.
Here is the output and derivative for a single
dimension, illustrating that this is indeed performing
the same operation as the ReLU.

\includegraphics[height=2in]{fs/output_2.pdf}
\includegraphics[height=2in]{fs/output_3.pdf}

\newpage
\textbf{The sigmoid.}
Recall from \cref{eq:sigmoid-proj} that the optimization view is
\begin{equation*}
f(x) = \argmin_{0<y<1} \;\; -x^\top y -H_b(y).
\end{equation*}
We can implement this layer with:
\begin{lstlisting}
x = cp.Parameter(n)
y = cp.Variable(n)
obj = cp.Minimize(-x.T*y - cp.sum(cp.entr(y) + cp.entr(1.-y)))
prob = cp.Problem(obj)
layer = CvxpyLayer(prob, params=[x], out_vars=[y])
\end{lstlisting}
We can also check that the output and derivative matches
what we expect from the usual sigmoid function:

\includegraphics[height=2in]{fs/output_6.pdf}
\includegraphics[height=2in]{fs/output_7.pdf}

\textbf{The softmax.}
Lastly recall from \cref{eq:simplex-proj} that the optimization view is
\begin{equation*}
f(x) = \argmin_{0<y<1} \;\; -x^\top y - H(y) \;\; \st\;\; 1^\top y = 1
\end{equation*}
We can implement this layer with:
\begin{lstlisting}
x = cp.Parameter(d)
y = cp.Variable(d)
obj = cp.Minimize(-x.T*y - cp.sum(cp.entr(y)))
cons = [sum(y) == 1.]
prob = cp.Problem(obj, cons)
layer = CvxpyLayer(prob, params=[x], out_vars=[y])
\end{lstlisting}

\newpage
\subsection{The OptNet QP}
We can re-implement the OptNet QP layer from \cref{sec:optnet}
with our differentiable \cvxpy{} layer in a few lines of code.
The OptNet layer is represented as a convex quadratic program
of the form
\begin{equation}
\begin{split}
x^\star = \argmin_{x} \;\; & \frac{1} {2}x^\top Qx + p^\top x \\
\subjectto \;\; & Ax = b \\
& Gx \leq h \\
\end{split}
\label{eq:cvxpy-qp}
\end{equation}
where $x \in \mathbb{R}^n$ is our optimization variable
$Q \in \mathbb {R}^{n \times n} \succeq 0$
(a positive semidefinite matrix),
$p \in \mathbb {R}^n$,
$A\in \mathbb{R}^{m \times n}$,
$b \in \mathbb{R}^m$,
$G \in \mathbb{R}^ {p \times n}$ and
$h \in \mathbb{R}^{p}$ are problem data.
We can implement this with:

\begin{lstlisting}
Q = cp.Parameter((n, n), PSD=True)
p = cp.Parameter(n)
A = cp.Parameter((m, n))
b = cp.Parameter(m)
G = cp.Parameter((p, n))
h = cp.Parameter(p)
x = cp.Variable(n)
obj = cp.Minimize(0.5*cp.quad_form(x, Q) + p.T * x)
cons = [A*x == b, G*x <= h]
prob = cp.Problem(obj, cons)
layer = CvxpyLayer(prob, params=[Q, p, A, b, G, h], out=[x])
\end{lstlisting}

This layer can then be used by passing in the
relevant parameter values:
\begin{lstlisting}
Lval = torch.randn(nx, nx, requires_grad=True)
Qval = Lval.t().mm(Lval)
pval = torch.randn(nx, requires_grad=True)
Aval = torch.randn(ncon_eq, nx, requires_grad=True)
bval = torch.randn(ncon_eq, requires_grad=True)
Gval = torch.randn(ncon_ineq, nx, requires_grad=True)
hval = torch.randn(ncon_ineq, requires_grad=True)
y = layer(Qval, pval, Aval, bval, Gval, hval)
\end{lstlisting}

\newpage
\subsection{Learning Polyhedral Constraints}
We demonstrate how gradient-based learning can be done with
a \cvxpy layer in this synthetic example.
Consider the polyhedrally constrained
projection problem
\begin{align*}
\hat y = \argmin_y\;\; &\frac{1}{2}||p-y||_2^2  \\
 {\rm s.t.}\;\; & Gy\leq h \\
\end{align*}
Suppose we don't know the polytope's parameters $\theta=\{G, h\}$
and want to learn them from data.
Then using the MSE for $\ell$, we can randomly initialize
ellipsoids $\theta$ and learn them with gradient steps $\nabla_\theta \ell$.
We note that this problem is meant for illustrative purposes and
could be solved by taking the convex hull of the input data points.
However our approach would still work if this was over a latent
and unobserved part of the model, of if you want to take an
approximate convex hull that limits the number of polytope edges.

We can implement this layer with the following code.
\cref{fig:polytope-results} shows the results of learning
on two examples.
Each problem has a true known polytope that we show in blue
and the model's approximation is in red.
Learning starts on the left with randomly initialized
polytopes that are updated with gradient steps,
which are shown in the images progressing to the right.

\begin{lstlisting}
G = cp.Parameter((m, n))
h = cp.Parameter(m)
p = cp.Parameter(n)
y = cp.Variable(n)
obj = cp.Minimize(0.5*cp.sum_squares(y-p))
cons = [G*y <= h]
prob = cp.Problem(obj, cons)
layer = CvxpyLayer(prob, params=[p, G, h], out=[y])
\end{lstlisting}

\begin{figure}[t]
  \centering
  \includegraphics[width=0.6\textwidth]{polytope-frames.png} \\
  \cblock{190}{201}{216} True Polytope \enskip
  \cblock{176}{128}{131} Approximate Polytope
  \caption{
    Learning polyhedrally constrained problems.
  }
  \label{fig:polytope-results}
\end{figure}

\newpage
\subsection{Learning Ellipsoidal Constraints}
In addition to learning polyhedral constraints, we can easily
learn any parameterized convex constraint set.
Suppose instead that we want to learn an ellipsoidal
projection of the form
\begin{align*}
\hat y = \argmin_y\;\; &\frac{1}{2}||p-y||_2^2  \\
 {\rm s.t.}\;\; & \frac{1}{2}(y-z)^\top A(y-z) \leq 1
\end{align*}
with ellipsoid parameters $\theta=\{A,z\}$
This is an interesting optimization problem to consider because
it is an example of doing learning with a non-polytope
cone program (a SOCP), which prior approaches such as OptNet
could not easily handle.

We can implement this layer with the following code.
\cref{fig:ellipsoid-results} visualizes the learning process
on two examples.
As before, each problem has a true known ellipsoid that we show in blue
and the model's approximation is in red.
Learning starts on the left with randomly initialized
ellipsoids that are updated with gradient steps,
which are shown in the images progressing to the right.

\begin{lstlisting}
A = cp.Parameter((n, n), PSD=True)
z = cp.Parameter(n)
p = cp.Parameter(n)
y = cp.Variable(n)
obj = cp.Minimize(0.5*cp.sum_squares(y-p))
cons = [0.5*cp.quad_form(y-z, A) <= 1]
prob = cp.Problem(obj, cons)
layer = CvxpyLayer(prob, params=[p, A, z], out=[y])
\end{lstlisting}

\begin{figure}[t]
  \centering
  \includegraphics[width=0.6\textwidth]{ellipsoid-frames.png} \\
  \cblock{190}{201}{216} True Ellipsoid \enskip
  \cblock{176}{128}{131} Approximate Ellipsoid
  \caption{
    Learning ellipsoidally constrained problems.
  }
  \label{fig:ellipsoid-results}
\end{figure}

\newpage
\section{Evaluation}
\label{sec:eval}

In this section we analyze the runtime of our layer's forward
and backward passes compared to hand-written implementations
for commonly used optimization layers.
We will focus on three tasks:

\paragraph{Task 1: Dense QP.}
We consider a QP layer of the form
\cref{eq:cvxpy-qp} with a dense quadratic objective
and dense inequality constraints.
Our default experiment uses a QP with 100 latent
variables, 100 inequality constraints, and
a minibatch size of 128 examples.
We chose this task to understand how the performance
of our \cvxpy{} layer compares to the \qpth implementation
from \cref{sec:optnet}, which we use as a comparison point.
The problem size we consider here is comparable
to the QP problem sizes considered in \cref{sec:optnet}.
The backwards pass of \qpth is optimized to use a single
batched, pre-factorized linear system solve.

\paragraph{Task 2: Box QP.}
We consider a QP layer of the form
\cref{eq:cvxpy-qp} with a dense quadratic objective
constrained to the box $[-1, 1]^n$.
Our default experiment uses a QP with 100 latent
variables and a minibatch size of 128 examples.
We chose this task to study the impacts of
sparsity on the runtime.
We again use \qpth as the comparison point for these experiments.

\paragraph{Task 3: Linear Quadratic Regulator (LQR).}
We consider a continuous-state-action, discrete-time, finite-horizon
LQR problem of the form
\begin{equation}
  \label{eq:cvxpyth:lqr}
  \tau^{\star}_{1:T} = \argmin_{\tau_{1:T}}\;\;
  \sum_{t} \frac{1}{2} \tau_t^\top  C_t \tau_t + c_t^\top  \tau_t \;\;
  \subjectto\;\;
  x_1 = \xinit,\
  x_{t+1} = F_t\tau_t + f_t.
\end{equation}
where $\tau_{1:T} = \{x_t, u_t\}_{1:T}$ is the nominal trajectory,
$T$ is the horizon,
$x_t, u_t$ are the state and control at time $t$,
$\{C_t, c_t\}$ parameterize a convex quadratic cost,
and $\{F_t, f_t\}$ parameterize an affine system
transition dynamics.
We consider a control problem with 10 states,
2 actions, and a horizon of 5.
We compare to the differentiable model
predictive control (MPC) solver from
\citep{amos2018differentiable}, which uses batched
PyTorch operations to solve a batch of LQR problems with
the Riccati equations, and then implements the backward
pass with another, simpler, LQR solve with
the Riccati equations.

For each of these tasks we have measured the forward
and backward pass execution times for our layer in
comparison to the specialized solvers.
We have run these experiments on an unloaded system with
an NVIDIA GeForce GTX 1080 Ti GPU and
a four-core 2.60GHz Intel Xeon E5-2623 CPUs hyper-threaded
to eight cores.
We set the number of OpenMP threads to 8 for our experiments.
For numerical stability, we use 64-bits for all of
our implementations and baselines.
For \qpth and our implementation, we use an iteration
stopping condition of $\epsilon=10^{-3}$.

\newpage
\subsection{Forward pass profiling}
\cref{fig:eval:fwd} summarizes our main forward pass execution
times. \cref{fig:eval:fwd:all} shows the runtimes of all of the
modes and batch sizes, and \cref{fig:eval:fwd:speedups}
illustrates the speedup of our best mode compared to the
specialized solvers.
We have implemented and run every mode from \cref{sec:cp:efficient}
and our summary presents the best-performing mode,
which in every case on the GPU is our block direct solver.
On the CPU, serializing SCS calls is competitive for
problems with more sparsity.
For dense and sparse QPs on the CPU and GPU, our batched
SCS+PyTorch direct cone solver is faster than
the \qpth solver, which likely comes from the
acceleration, convergence, and normalization
tricks in SCS that are not present in \qpth.
The LQR task presents a sparse problem that illustrates
the challenges to using a general cone program formulation.
Our specialized solver that solves the Riccati equations in
batched form exploits the sparsity pattern of the problem
that is extremely difficult for the general cone program
formulation we consider here to take advantage of.
If the correct mappings to the cone program exist,
our layer could be modified to accept an
optimized user-provided solver for the forward pass
so that users can still take advantage of our backward
pass implementation.

\begin{figure}[t]
  \centering
  \includegraphics[width=0.9\textwidth]{prof/forward-summary.pdf} \\
  \cblock{128}{128}{128} Specialized Solver \enskip
  \cblock{208}{46}{47} PyTorch Block Direct \enskip
  \cblock{86}{165}{84} SCS Serial Direct
  \caption{
    Forward pass execution times.
    For each task we run ten trials
    on an unloaded system and normalize the runtimes to the
    CPU execution time of the specialized solver.
    The bars show the 95\% confidence interval.
    For our method, we show the best performing mode.
  }
  \label{fig:eval:fwd}
\end{figure}

\begin{figure}[!h]
  \centering
  \includegraphics[width=0.9\textwidth]{prof/CPU-forward-speedups.pdf}
  \includegraphics[width=0.9\textwidth]{prof/CUDA-forward-speedups.pdf}

  \cblock{208}{46}{47} PyTorch Block Direct \enskip
  \cblock{86}{165}{84} SCS Serial Direct \enskip
  \cblock{230}{127}{25} SCS Block Direct

  \caption{
    Forward pass execution time speedups of our best
    performing method in comparison to the specialized
    solver's execution time.
    For each task we run ten trials on an unloaded system.
    The bars show the 95\% confidence interval.
  }
  \label{fig:eval:fwd:speedups}
\end{figure}

\begin{figure}[!h]
  \centering
  \includegraphics[width=0.9\textwidth]{prof/prof_qp_dense-forward.pdf}
  \includegraphics[width=0.9\textwidth]{prof/prof_qp_box-forward.pdf}
  \includegraphics[width=0.9\textwidth]{prof/prof_mpc-forward.pdf}

  \cblock{128}{128}{128} Specialized Solver \enskip
  \cblock{208}{46}{47} PyTorch Block Direct \enskip
  \cblock{68}{125}{171} PyTorch Block Indirect \\
  \cblock{86}{165}{84} SCS Serial Direct \enskip
  \cblock{146}{86}{155} SCS Serial Indirect \enskip
  \cblock{230}{127}{25} SCS Block Direct \enskip
  \cblock{235}{235}{71} SCS Block Indirect

  \caption{Full data for the forward pass execution times.
    For each task we run ten trials on an unloaded system.
    The bars show the 95\% confidence interval.
  }
  \label{fig:eval:fwd:all}
\end{figure}

\newpage~\newpage~\newpage
\subsection{Backward pass profiling}
\label{sec:cvxpyth:bw_prof}
In this section we compare the backward pass times of
our layer in comparison to the specialized solvers
on the same three tasks as before: the dense QP,
the box QP, and LQR.
We show that differentiating through our conic
solver is competitive in comparison to the
specialized solver.
As a comparison point, the \qpth solver exploits
the property that the linear system for the backward pass is the
same as the linear system in the forward pass and can therefore
do the backward pass with a single pre-factorized solve.
The LQR solver exploits the property that the backward
pass for LQR can be interpreted as another LQR problem that
can efficiently be solved with the Riccati recursion.

These comparisons are important because the linear system for
differentiating cone programs in \cref{eq:cvxpyth:diff_efficient}
is a more general form and cannot leverage the same exploits
as the specialized solvers.
To get an intuition of what these linear systems look like on
our tasks, we plot sample maps on smaller problems of the
coefficient matrix in \cref{fig:cvxpyth:sample_maps}.
This illustrates the sparsity that is typically present in the
linear system that needs to be solved, but also illustrates
that beyond sparsity, there is no other common property that
can be exploited between the tasks.

\begin{figure}[!t]
  \centering
  \includegraphics[height=2.0in]{Ms_qpth_dense.pdf}
  \includegraphics[height=2.0in]{Ms_qpth_box.pdf}
  \includegraphics[height=2.0in]{Ms_MPC.pdf}
  \caption{Sample linear system coefficients for
    the backward pass system in \cref{eq:cvxpyth:diff_efficient}
    on smaller versions of the tasks we consider.
    The tasks we consider are approximately five
    times larger than these systems.
  }
  \label{fig:cvxpyth:sample_maps}
\end{figure}

To understand how many LSQR iterations are necessary
to solve our task, we compare the approximate derivatives computed
by LSQR to the derivatives obtained by directly solving
the linear system in \cref{fig:cvxpyth:lsqr_conv}.
This shows that typically 500-1000 LSQR iterations
are necessary for the tasks that we consider.
In some cases such as $\partial x^\star / \partial A$
for LQR, the approximate gradient computed by LSQR does
never converges exactly to the true gradient.

\begin{figure}[!t]
  \centering
  \includegraphics[width=0.9\textwidth]{lsqr_dense_qp.pdf}
  \includegraphics[width=0.9\textwidth]{lsqr_box_qp.pdf}
  \includegraphics[width=0.9\textwidth]{lsqr_mpc.pdf}

  \cblock{68}{125}{171} $\partial x^\star/\partial A$ \enskip
  \cblock{208}{46}{47} $\partial x^\star / \partial b$ \enskip
  \cblock{146}{86}{155} $\partial x^\star / \partial c$
  \caption{LSQR convergence for the backward pass systems.
    The shaded areas show the 95\% confidence interval
    across ten problem instances.
  }
  \label{fig:cvxpyth:lsqr_conv}
\end{figure}

\begin{figure}[!t]
  \centering
  \includegraphics[width=0.9\textwidth]{prof/prof_qp_dense-backward.pdf}
  \includegraphics[width=0.9\textwidth]{prof/prof_qp_box-backward.pdf}
  \includegraphics[width=0.9\textwidth]{prof/prof_mpc-backward.pdf}

  \cblock{128}{128}{128} Specialized Solver \enskip
  \cblock{208}{46}{47} Direct (Dense) \enskip
  LSQR
  (\cblock{68}{125}{171} 100 \cblock{86}{165}{84} 500 \cblock{146}{86}{155} 1000)
  Iterations
  \caption{
    Backward pass execution times.
    For each task we run ten trials on an unloaded system.
    The bars show the 95\% confidence interval.
  }
  \label{fig:cvxpyth:bw}
\end{figure}

\newpage~\newpage~\newpage
\cref{fig:cvxpyth:bw} compares our backward pass times to
the specialized solvers for the QP and LQR tasks.
This shows that there is a slight computational overhead
in comparison to the specialized solvers, but that
solving the linear system is still tractable for these tasks.
The LSQR runtime is serialized across the batch, and is
currently only implemented on the CPU.
We emphasize that if the backward pass time becomes a bottleneck,
the runtime can be further improved by further exploiting the
sparsity by investigating other direct and indirect solvers
for the systems, or by exploiting the property that we mentioned
earlier in \cref{sec:diff-cp} that for simple cones like the
free and non-negative cones, parts of the system become the same as
parts of the KKT system.

\section{Conclusion}
This section has presented a way of differentiating through
cone programs that enabled us to create a powerful prototyping
tool for differentiable convex optimization layers.
Practitioners can use this library in place of hand-implementing
a solver and implicitly differentiating the KKT conditions.
The speed of our tool is competitive with the speed of specialized
solvers, even in the batched setting required for machine learning.

%%% Local Variables:
%%% coding: utf-8
%%% mode: latex
%%% TeX-engine: xetex
%%% TeX-master: "../thesis"
%%% End:

\part{Conclusions and Future Directions}
\chapter{Conclusions and Future Directions}
\label{sec:conclusions}

In this thesis we have introduced new building blocks and
fundamental components for machine learning that enable
optimization-based domain knowledge to be injected
into the modeling pipeline.
We have presented the \emph{OptNet} architecture as a
foundation for convex optimization layers and the
\emph{input-convex neural network} architecture as a
foundation for deep convex energy-based learning.
We have shown how these techniques can be applied to
differentiable model-predictive control and
top-$k$ learning.
Differentiable optimization-based modeling components
provide an expressive set of operations and
have a promising set of future directions.
To enable rapid prototyping in this space, we have shown
how \cvxpy can be turned into a differentiable layer.
In the following we discuss areas where optimization-based
has been applied and has potential to continue making
an impact.

\begin{itemize}
\item \textbf{Game theory.}
  The game theory literature typically focuses on finding
  optimal strategies of playing games with known rules.
  While the rules of a lot of games are known explicitly,
  scenarios could come up where it's useful to learn the
  rules of a game and to have a ``game theory''
  equilibrium-finding layer.
  For example in reinforcement learning, an agent can have an
  arbitrary differentiable ``game theory'' layer that is able
  of representing complex tasks, state spaces, and
  problems as an equilibrium-finding problem in a game
  where the rules are automatically extracted.
  This has started to be explored in
  \citet{ling2018game}.
\item \textbf{Stochastic optimization and end-to-end learning.}
  Typically probabilistic models are used in the context of
  larger systems. When these systems have an objective
  that is being optimized, it is usually ideal to incorporate
  the knowledge of this objective into the probabilistic modeling
  component.
  If the downstream systems involve solving
  stochastic optimization problems, as in power-systems,
  creating an end-to-end differentiable architecture is
  more difficult and can be done by using
  differentiable optimization as in \citet{donti2017task}.
\item \textbf{Reinforcement learning and control.}
  \begin{itemize}
  \item \textbf{Safety.} RL agents may be deployed in scenarios when
    the agent should avoid parts of the state space,
    \eg in safety-critical environments.
    Differentiable optimization layers can be used
    to help constrain the policy class so that these
    regions are avoided.
    This is starting to be explored in
    \citet{dalal2018safe,pham2018optlayer}.
  \item \textbf{Differentiable control and planning.}
    The differentiable MPC approach we presented in \cref{sec:empc}
    is just one step towards a significantly broader vision
    of integrating control and learning for doing imitation
    or policy learning.
  \item \textbf{Physics-based modeling.}
    When RL environments involve physical systems,
    it may be useful to have a physics-based modeling.
    This can be done with a differentiable
    physics engines as in \citet{de2018end}.
  \item \textbf{Inverse cost and reward learning.}
    In many multi-agent control scenarios modeling agents
    as controllers that are optimizing a control objective
    is a powerful paradigm \citep{ng2000algorithms,finn2016guided}.
    TODO
    Differentiable controllers
    are useful when trying to reconstruct an optimization
    problem that other agents are solving.
    This is done in the context of cost shaping in
    \citet{tamar2017learning}.
  \item \textbf{Multi-agent environments.} TODO
  \item \textbf{Control in high-dimensional state spaces.} TODO
  % \item Universal planning networks: \citet{srinivas2018universal}
  \end{itemize}
\item \textbf{Discrete, combinatorial, and submodular optimization.}
  The space of discrete, combinatorial, and mixed optimization problems
  captures an even more expressive set of operations than
  the continuous and convex optimization problems we have considered
  in this thesis.
  Similar optimization components can be made for some of these
  types of problems, as explored in
  \citet{djolonga2017differentiable,tschiatschek2018differentiable,mensch2018differentiable,niculae2018sparsemap,niculae2017regularized}.
\item \textbf{Meta-learning.}
  Some meta-learning formulations such as \citet{finn2017model}
  and \citet{ravi2016optimization}
  involve learning through an unrolled optimizer that
  typically solve an unconstrained, continuous, and
  non-convex optimization problem.
  In some cases, unrolling through a solver with
  many iterations may require inefficient amounts
  of compute or memory.
  Meta-learning methods could be improved by using a differentiable
  closed-form solver as
  \citet{bertinetto2018meta} explores.
\item \textbf{Optimization viewpoints of standard components.}
  A motivation behind this thesis work has been the optimization
  viewpoints of standard layers we discussed in \cref{sec:bg:existing}.
  Many other directions can be taken with the viewpoints, such
  as the proximal operator viewpoint in \citet{bibi2018deep}
  that interprets deep layers as stochastic solvers.
\end{itemize}

%%% Local Variables:
%%% coding: utf-8
%%% mode: latex
%%% TeX-engine: xetex
%%% TeX-master: "../thesis"
%%% End:

\chapter*{Bibliography}
\addcontentsline{toc}{chapter}{Bibliography}

\vspace{-25mm}
This bibliography contains \total{citenum} references.
\vspace{10mm}

\printbibliography[heading=none]

\end{document}

%%% Local Variables:
%%% coding: utf-8
%%% mode: latex
%%% TeX-engine: xetex
%%% End:
